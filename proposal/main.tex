
\documentclass[a4paper]{article}

%\usepackage[pages=all, color=black, position={current page.south}, placement=bottom, scale=1, opacity=1, vshift=5mm]{background}

\usepackage[margin=1in]{geometry} % full-width

% AMS Packages
\usepackage{amsmath}
\usepackage{amsthm}
\usepackage{mathtools}
\usepackage{amssymb}
\usepackage{booktabs}

% Unicode
\usepackage[utf8]{inputenc}
\usepackage{hyperref}
\newcommand{\AL}[1]{{\color{blue}{[Andrew: #1]}}}

\hypersetup{
	unicode,
%	colorlinks,
%	breaklinks,
%	urlcolor=cyan, 
%	linkcolor=blue, 
	pdfauthor={Author One, Author Two, Author Three},
	pdftitle={A simple article template},
	pdfsubject={A simple article template},
	pdfkeywords={article, template, simple},
	pdfproducer={LaTeX},
	pdfcreator={pdflatex}
}

% Vietnamese
%\usepackage{vntex}

% Natbib
%\usepackage[sort&compress,numbers,square]{natbib}
%\bibliographystyle{mplainnat}

% Theorem, Lemma, etc
\theoremstyle{plain}
\newtheorem{theorem}{Theorem}
\newtheorem{corollary}[theorem]{Corollary}
\newtheorem{lemma}[theorem]{Lemma}
\newtheorem{claim}{Claim}[theorem]
\newtheorem{axiom}[theorem]{Axiom}
\newtheorem{conjecture}[theorem]{Conjecture}
\newtheorem{fact}[theorem]{Fact}
\newtheorem{hypothesis}[theorem]{Hypothesis}
\newtheorem{assumption}[theorem]{Assumption}
\newtheorem{proposition}[theorem]{Proposition}
\newtheorem{criterion}[theorem]{Criterion}
\theoremstyle{definition}
\newtheorem{definition}[theorem]{Definition}
\newtheorem{example}[theorem]{Example}
\newtheorem{remark}[theorem]{Remark}
\newtheorem{problem}[theorem]{Problem}
\newtheorem{principle}[theorem]{Principle}

\usepackage{graphicx, color}
\graphicspath{{fig/}}

%\usepackage[linesnumbered,ruled,vlined,commentsnumbered]{algorithm2e} % use algorithm2e for typesetting algorithms
\usepackage{algorithm, algpseudocode} % use algorithm and algorithmicx for typesetting algorithms
\usepackage{mathrsfs} % for \mathscr command

\usepackage{lipsum}

% Author info
\title{CSC2611 Project Proposal}
\author{Yuan-Hong Liao}

\date{
	Computer Science at the University of Toronto \\ \texttt{andrew@cs.toronto.edu}
}

\begin{document}
\maketitle

\section{Proposal}

\subsection{Motivation}
\AL{Some overview.} In this project, we want to improve the performance of lexical semantic change detection (LSC).
\AL{Emphasize unsupervised}

\subsection{Problem Formulation}
Given two time-specific corpora $C^1, C^2$ and $N$ target words $\{w_i\}_{i=1:N}$.
We denote the set of sentences containing the target word $w$ in corpus $C^1$ as $S_w^1$.
We perform unsupervised predictions on which words lost or gained sense(s) between $C^1$ and $C^2$, and which ones did not.
We formulate the problem as a binary classification problem over each target word and report the accuracy.

\subsection{Proposed Approach: Influence Function}

We following the intuition that if a word sharing similar \textit{semantic meanings} between two different corpora, the sentences containing the word also have similar \textit{influences} to a model.
On the contrary, if a word losts or gains any sense between two corpora, the sentences containing the word impact a model differently.

To measure the influrnence to a model, let us the empirical risk minimizer $\hat{\theta} \coloneqq \arg\min_{\theta \in \Theta} \mathcal{L} (s, \theta) $

For a target word $w$ and a set of sentences containing the target word in two different corpora $S_w^1 \subseteq C^1$ and $S_w^2 \subseteq C^2$ , we train two different models that excluding the two sets of sentences separately:

\begin{align}
\begin{split}
	\theta_1 & \leftarrow \arg\min \mathcal{L}(\{C^1 \setminus S_w^1, C^2\}) \\
	\theta_2 & \leftarrow \arg\min \mathcal{L}(\{C^2 \setminus S_w^2, C^1\})
\end{split}
\end{align}



If the target word is stable across two corpora, we expect $\theta_1$ and $\theta_2$ behaves similarly, and vice versa.
However, it is prehibitive to train two neural networks to detect LSC of a single target word. 




%\noindent\textbf{Notations.}



\subsection{Datasets}



\AL{Find potential datasets to evaluate semantic change detection}
\AL{Given that there are some semantic changes in the words, can we ``see'' it? How to evaluate or compare?}
\AL{What can we do when we know that a word undergo a semantic change through time}
\AL{Can we input masked language to the CLIP model? so that we can see the diachronich change of the missing word}

\end{document}