% This must be in the first 5 lines to tell arXiv to use pdfLaTeX, which is strongly recommended.
\pdfoutput=1
% In particular, the hyperref package requires pdfLaTeX in order to break URLs across lines.

\documentclass[11pt]{article}

% Remove the "review" option to generate the final version.
%\usepackage[review]{acl}
\usepackage[]{acl}

% Standard package includes
\usepackage{times}
\usepackage{latexsym}

% For proper rendering and hyphenation of words containing Latin characters (including in bib files)
\usepackage[T1]{fontenc}
% For Vietnamese characters
% \usepackage[T5]{fontenc}
% See https://www.latex-project.org/help/documentation/encguide.pdf for other character sets

% This assumes your files are encoded as UTF8
\usepackage[utf8]{inputenc}

% This is not strictly necessary, and may be commented out,
% but it will improve the layout of the manuscript,
% and will typically save some space.
\usepackage{microtype}

\usepackage[dvipsnames]{xcolor}
\newcommand{\AL}[1]{{\color{blue}{[Andrew: #1]}}}

% If the title and author information does not fit in the area allocated, uncomment the following
%
%\setlength\titlebox{<dim>}
%
% and set <dim> to something 5cm or larger.

\title{Unsupervised Lexical Semantic Change Detection via Influences Function}

% Author information can be set in various styles:
% For several authors from the same institution:
% \author{Author 1 \and ... \and Author n \\
%         Address line \\ ... \\ Address line}
% if the names do not fit well on one line use
%         Author 1 \\ {\bf Author 2} \\ ... \\ {\bf Author n} \\
% For authors from different institutions:
% \author{Author 1 \\ Address line \\  ... \\ Address line
%         \And  ... \And
%         Author n \\ Address line \\ ... \\ Address line}
% To start a seperate ``row'' of authors use \AND, as in
% \author{Author 1 \\ Address line \\  ... \\ Address line
%         \AND
%         Author 2 \\ Address line \\ ... \\ Address line \And
%         Author 3 \\ Address line \\ ... \\ Address line}

\author{
  Yuan-Hong Liao \\
  University of Toronto \\
  \texttt{andrew@cs.toronto.edu}}

\begin{document}
\maketitle
\begin{abstract}
  Languages change through time and detecting lexical semantic change in languages provides better understanding across different cultures, regions, and time spans.
  Determining the ground truth of languages semantic change requires high linguistics expertise. 
  Therefore, this work aim to detect the lexical semantic change between two corpora in an unsupervised manner.
  This work leverages the fact that language models learn the word usage in a given corpora. 
  Once a word undergoes any semantic change, the language models trained on the corpus change accordingly.
  I implement this motivation via influence functions to efficiently approximate the model trained on different corpus configuratinos.
  In the SemEval2020 benchmark, the proposed approach, InfDetect, achieves ?? accuracy in the binary task, outperforming ?? from 22 teams in the SemEval2020 chanllenges.
  In the ranking task, InfDetect achieves spearman correlation ??, outperforming ?? from 22 teams in the SemEval2020 chanllenges.
\end{abstract}

\section{Introduction}

The meaning of words continuously changes over time, reflecting complicated processes in language and society. 
For example, the meaning of ``bubble'' extends to the travel bubble and social bubble due to the Covid-19 pandemic.
Subtle shifts or cultural associations also impact the perpetual meaning of the word ``Iraq'' and ``Syria''.
Studying these types of changes in meaning enables researchers to learn more about human language and extract temporal-dependent data from texts.

The availability of large corpora and the progress in natural languages processing enables us to dissect the lexical semantic change (LSC) problem from a computational perspective~\cite{diachronic-survey}.
Due to the ill-defined characteristic in LSC, the size of gold standards is very limited.
Additionally, the need for robust evaluation plays a crucial role in LSC detection. 
The level of increasing level of abstraction is often ignored in evaluation. 
So far, semantic annotation is the only way to evaluate methods on historical corpora while making sure that expected changes are present in the text. 
Annotating involves a significant investment of time and funds and results in a limited test set~\cite{challenges_lsc}.


In this work, I focus on detecting LSC in an unsupervised manner in the SemEval2020 benchmark~\cite{semeval2020}, the first larger-scale, an openly available dataset with high-quality, hand-labeled judgments.
To approach the LSC problem, I leverage the progress in ML and NLP by analyzing the behavior of a trained language model.
The idea is that I expect \textit{if a word has experienced a lexical semantic change between two time periods, removing the training data containing the word from a time period changes the network behaviors a lot}.
Vice versa, if the word contains similar meanings across two corpora, removing one of the corpora from the training data does not change the network too much.

I implement the idea with influence functions~\cite{influence_fn} -- a classic technique from robust statistics -- to trace a model's predictions through the learning algorithm and back to its training data.
The proposed approach achieves 0.7 accuracies in the binary task, outperforming 20 from 22 teams in the SemEval2020 challenges.
In the ranking task, it achieves spearman's correlation of 0.59, outperforming all 22 teams in the SemEval2020 challenges.


\section{Related Work}


\subsection{Unsupervised Lexical Semantic Change Detection}




There are two strands of approach for LSC detection in the computational field.
The first strand deals with words as a whole and determines change based on a word's dominant sense~\cite{dominant-sense1,dominant-sense2}.
The second strand deals with a word's sense individually~\cite{second-strand1,second-strand2}.
These two correspond to two tasks in SemEval2020~\cite{semeval2020}: binary classification and ranking problem.
Prior work~\cite{temporalteller} leverages synchronic word embeddings and computes the distance between two corpora.
On the contrary, ~\cite{diachronic-survey} surveys approaches that use diachronic word embeddings to perform semantic shifts detection.
In this work, I go from a different direction: I investigate the model's changes w.r.t. the target words by utilizing influence functions~\cite{influence_fn}.


\section{Problem Formulation}


\section{InfDetect}

\subsection{Influence Function}

\subsection{Detect Semantic Change via Data Influences}


\section{Experiments}

\subsection{SevEval2020}

I use the English task in SemEval2020 benchmark~\cite{semeval2020}.
The corpus is from Clean Corpus of Historical American English (CCOHA)~\cite{ccoha1,ccoha2}, which spans 1810s-2000s.
The corpus is split into two time-specific corpora $C_1, C_2$, as defined in Table~\ref{tab:cchoa}.
There are, in total, 37 target words, and I will evaluate the proposed method in both binary classification and ranking problems.


\noindent\textbf{Metrics.} 
For classification, we use standard accuracy. For ranking problem, we use spearman's correlation.

\begin{table}[t]
\centering
\resizebox{0.8\linewidth}{!}{%
\begin{tabular}{@{}crrrr@{}}
\toprule
\multicolumn{1}{r}{}       & \multicolumn{1}{c}{periods} & \multicolumn{1}{c}{tokens} & \multicolumn{1}{c}{types} & \multicolumn{1}{c}{TTR} \\ \midrule
\multicolumn{1}{c|}{$C_1$} & 1810–1860                   & 6.5M                       & 87k                       & 13.38                   \\
\multicolumn{1}{c|}{$C_2$} & 1960-2010                   & 6.7M                       & 150k                      & 22.38                   \\ \bottomrule
\end{tabular}%
}
\caption{Statistics of test corpora. TTR = Type-Token ratio (number of types  / number of tokens * 1000)}
\label{tab:cchoa}
\end{table}

%\AL{scale of data, where does the data come from, how do they obtain the ground truth, what is the sub-tasks, evaluation metrics}

\subsection{Implementation Details}
For the empirical minimizer, we use a modified BERT~\cite{BERT} model.
In my case, I do not necessarily need a model to achieve the SOTA language modeling performance. 
To make the training less computationally intensive, I decrease the hidden units from 768 to 128, 12 attention heads to 8 attention heads, and set the max length to 128.
The modification reduces the model size from 110M to 10M parameters.
To approximate the hessian inverse product in Eq.~\ref{eq:scores_mlm}, we set the scale to be 5000, recursion depth 10000, and damping $1e-2$.
The source code is released at https://github.com/andrewliao11/csc2611-course-semantic-change.


\subsection{Binary Classificaiton}

For binary classification, I achieve 0.7 accuracies, outperforming 20 out of 22 participants in SemEval challenge~\cite{semeval2020}.
Since each team can submit ten trials in the challenge, I try ten different thresholds that equally split the range between maximum influence and min influence and report the best performance.
I show the predicted influences in Fig.~\ref{fig:binary}.
For the words that do not experience any LSC change, the word ``ball'' has the greatest predicted influence. 
From the ground truth rank, the word ``ball'' is the 5th highest word, with a score of $0.409$.
The binary classification considers if the target word gain or loss any sense between two corpora, which is different from how the ground truth of the rank is measured.
I will discuss the comparison between binary classification and ranking problem in Sec.~\ref{sec:}


\begin{figure}[t]
\centering
\includegraphics[width=0.5\textwidth]{../project/src/influences_binary.pdf}
\caption{\textbf{Predicted Influences.} The red dash lines are the threshold for binary classification.}
\label{fig:binary}
\end{figure}



\subsection{Ranking}


For the ranking problem, I achieve $0.59$ spearman's correlation, outperforming all 22 participants in SemEval2020~\cite{semeval2020}.
The predicted influences and the ground truth grades are shown in Fig.~\ref{fig:rank}.
One failure case is the word ``plane'', which has a ground truth grade of 0.88, while the predicted influence is as low as 0.02, which ranks middle (19th) in terms of predicted influence.



\begin{figure}[t]
\centering
\includegraphics[width=0.5\textwidth]{../project/src/influences_grade_scatter.pdf}
\caption{\textbf{Predicted Influences vs. ground truth grade.}}
\label{fig:rank}
\end{figure}



%\subsection{Qualitative Results}

\subsection{Analysis}
The estimated LSC scores are the average predicted influences. 
In Fig.~\ref{fig:inf_time}, I show the predicted influences in $C_1$ and $C_2$. 
The word ``player'', ``fiction'', and ``relationship'' have the most difference between predicted influences in $C_1$ and $C_2$, more than 300\%.
For certain words, the computation of inverse HVP diverges. In these cases, I treat them as an outlier and do not consider their values.


\begin{figure}[t]
\centering
\includegraphics[width=0.5\textwidth]{../project/src/influences_by_corpus.pdf}
\caption{\textbf{Predicted Influences in different time periods.}}
\label{fig:inf_time}
\end{figure}

\section{Discussion}

The proposed approach shows that the influences function is highly correlated with LCS, achieving 0.7 in binary classification and 0.59 in spearman's correlation in the ranking problem.
However, the estimation of the influence function is time-consuming and not stable.
For each term in Eq.~\ref{eq:influence_fn_mlm}, it takes more than 20 minuites to compute.
The bottleneck is the approximation of inverse HVP. 
The approximation is done recursively, so it's not parallelizable. In our case, it takes 10000 recursion steps to reach convergence.
Additionally, the computation is not stable for some words (e.g., ``trump''), leading to divergence.
The computational expensive and instability hamper it from performing LCS in a continuous fashion. For example, monitoring the LCS for an extended period will be prohibitively computationally expensive.



% Entries for the entire Anthology, followed by custom entries
\bibliography{anthology,custom}

%\appendix

%\section{Example Appendix}
%\label{sec:appendix}

%This is an appendix.

\end{document}
